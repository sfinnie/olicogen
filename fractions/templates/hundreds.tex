% ---------------------------------------------------------------
% Document configuration setup.
%
% Sets up things like document format, page size and the libraries to use later
% ---------------------------------------------------------------

\documentclass[11pt, oneside]{article}   	% use "amsart" instead of "article" for AMSLaTeX format
\usepackage{geometry}                		% See geometry.pdf to learn the layout options. There are lots.
\geometry{letterpaper}                   		% ... or a4paper or a5paper or ...
%\geometry{landscape}                		% Activate for for rotated page geometry
%\usepackage[parfill]{parskip}    		% Activate to begin paragraphs with an empty line rather than an indent
\usepackage{graphicx}				% Use pdf, png, jpg, or eps§ with pdflatex; use eps in DVI mode
								% TeX will automatically convert eps --> pdf in pdflatex
\usepackage{amssymb}
\usepackage{tikz}
\usepackage{amsmath}
\usetikzlibrary{arrows}
%\title{Brief Article}
%\author{The Author}
%\date{}							% Activate to display a given date or no date

% ---------------------------------------------------------------
% Start of the document itself
% ---------------------------------------------------------------
\begin{document}
%\maketitle

% ---------------------------------------------------------------
% tikz is the LaTeX library for drawing 2D figures.  Use the
% arrows library for drawing the number lines
% ---------------------------------------------------------------
\usetikzlibrary{arrows}
% ---------------------------------------------------------------
% This "document" is really a program itself.  At the top level,
% it's several loops that draw variations of each diagram; e.g.
% the first loop counts from 0..19 (1..20 in normal people's
% language), drawing number lines for hundredths.
% We're creating 20 different drawings, which will be named e.g.
% "Frac1 b6 diagram hundredths1", "Frac1 b6 diagram hundredths2",
% etc.  Note the "y+1" in the name so the diagrams are numbered
% 1..20 instead of 0..19.
%
% Note: the double backslashes (i.e. '\\') are significant.  They
% mean "treat this line as a comment".  Otherwise, latex thinks its
% empty lines that need to be written into the document.
% ---------------------------------------------------------------
\section{Hundredths}
%\subsection{}
\foreach \y in {0,1,2,3,4}
{ Frac1 b6 diagram hundredths\the\numexpr\y+1
\\
\\
% ---------------------------------------------------------------
% Start drawing the pictures, beginning with the number line.
% ---------------------------------------------------------------
\begin{tikzpicture} [>=triangle 45, scale = 0.1]%use to change style of arrow head and scale of pic
\foreach \x in  {0,..., 107}%this tells you at which points to draw the lines and labels
\draw[shift={(\x,0)},color=black] (0pt,20pt) -- (0pt,-20pt);%this tells you the vert lines color and size
\foreach \x in  {0,..., 99}
\draw[shift={(\x,1.8)}] {rectangle(1,5.8)};
\foreach \z in  {0,10,20,30,40,50,60,70,80,90,100}%this tells you at which points to draw the lines and labels
\draw[shift={(\z,0)},color=black] (0pt,40pt) -- (0pt,-40pt);%this tells you the vert lines color and size
\draw[very thick, ->] (0,0) -- (110,0) ;
%\foreach \z in {0, 100}
%\draw[shift={(\z,0.3)}, color=black] {rectangle(1,0.7)};
\foreach \x in {0,100}
\draw[shift={(\x,0)},color=black] (0pt,0pt) -- (0pt,-12pt) node[below]{\LARGE\the\numexpr\x/100\relax}; %this tells you the labels under the vert lines
\foreach \x in {0,...,\y}
\draw[shift={(\x,1.8)}, fill=red] {rectangle(1,5.8)};

\end{tikzpicture}
\\}

\end{document}